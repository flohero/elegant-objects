%! Author = florian
%! Date = 30.10.22

% Preamble
\documentclass[11pt]{article}

% Packages
\usepackage{amsmath}
\usepackage[english]{babel}
\usepackage{url}
\usepackage{libertine}
\usepackage{libertinust1math}
\usepackage[T1]{fontenc}
\usepackage{hyperref}
\usepackage{glossaries}
\usepackage[a4paper, total={6in, 9in}]{geometry}

%Glossary
\makeglossaries
\newglossaryentry{EO}{
    name=elegant objects,
    description={An OOP paradigm}
}

\newglossaryentry{OOP}{
    name=object orientated programming,
    description={A programming paradigm that is centered around objects}
}

\title{Elegant Objects: A OOP Paradigm}
\author{Florian Weingartshofer}
\date{November 3, 2022}

% Document
\begin{document}
    \selectlanguage{english}
    \maketitle


    \section{Introduction}\label{sec:motivation}
    \Gls{EO} is an object orientated programming paradigm, with some extreme viewpoints on traditional methods.
    For example, null references or static methods are taboo.
    A cleaner, more maintainable and more readable codebase should be achieved by renouncing those traditional software development methods.
    \Gls{EO} also forces the programmer to think purely in objects and \gls{OOP} and not other programming paradigms, like procedural.



    \section{Content of the Paper}\label{sec:content-of-the-paper}
    The paper will first describe the twelve principles of \gls{EO} and show how they work by using some examples of the project, that is developed while writing the paper.

    Furthermore, will the paper draw conclusion on how and if the codebase is more maintainable.

    Lastly, will the paper describe hwo the developer experience changes by using \gls{EO}.
    Most important aspects are:
    \begin{itemize}
        \item Developer productivity
        \item Code readability
        \item Ease of understanding \gls{EO} concepts
    \end{itemize}


    \section{Project}\label{sec:project}
    The project will use the spotify web api\footnote{\url{https://developer.spotify.com/documentation/web-api/}} to gather song data.
    The data is then processed, saved to a database and then will be shown in a web interface.
    The whole application should be developed using \gls{EO} and libraries that respect EO principles.
    If it is not possible to use only EO libraries, it should be wrapped in a way, that the resulting objects use \gls{EO}.

    \section{Organization}\label{sec:organization}
    This and following documents are written in english.
    The project and the paper will be published on GitHub under an open source license.

    \printglossary

\end{document}