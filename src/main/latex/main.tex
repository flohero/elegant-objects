%! Author = florian
%! Date = 30.10.22

% Preamble
\documentclass[10pt,journal]{IEEEtran}

% Packages
\usepackage{amsmath}
\usepackage[english]{babel}
\usepackage{url}
\usepackage{libertine}
\usepackage{libertinust1math}
\usepackage[T1]{fontenc}
\usepackage{hyperref}
\usepackage{glossaries}
\usepackage[a4paper, total={6in, 9in}]{geometry}

%Glossary
\makeglossaries
\newglossaryentry{EO}{
    name=EO,
    description={An OOP paradigm}
}

\newglossaryentry{OOP}{
    name=OOP,
    description={A programming paradigm that is centered around objects}
}

\newglossaryentry{API}{
    name=API,
    description={An application programming interface
    allows two applications or a programmer and a library to talk to each other},
    plural={APIs}
}

\title{Elegant Objects: A OOP Paradigm}
\author{Florian Weingartshofer}
\date{November 3, 2022}

% Document
\begin{document}
    \maketitle

    \selectlanguage{english}


    \section{Introduction}\label{sec:introduction}


    \section{Principles}\label{sec:principles}
    The elegant objects paradigm is based on eleven principles.
    They will be discussed in this chapter, along with some examples and some limitations of the principles.

    \subsection{No Null}\label{subsec:no-null}
    Null references has been described as the billion-dollar mistake multiple times.



%    \bibliography{main}
%    \bibliographystyle{plain}

\end{document}
