%! Author = florian
%! Date = 11.12.22

% Preamble
One issue of the no null principle is that the author suggests the usage of empty objects, like a nobody employee as seen in figure\ \ref{fig:empty-object}.
In the companion blog\footnote{\url{https://www.yegor256.com/2014/05/13/why-null-is-bad.html}} the author writes how null will lead to slowly dying code, instead of failing fast.
He neglects that the usage of these empty objects can also lead to slowly dying code.
It appears like an operation was successful, but when calling specific methods it will result in an exception.
If null is not an acceptable return value, it should result in an exception and not return an empty object.
Otherwise, this pattern will result in code that checks if an object is the same as an empty object.
In other words, more complicated null checks.

\begin{figure}[h]
    \caption{Empty Object}
    \lstinputlisting[language=Java,basicstyle=\tiny,label={lst:empty-object}]{assets/code/Employee.java}\label{fig:empty-object}
\end{figure}