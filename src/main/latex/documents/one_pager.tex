%! Author = florian
%! Date = 30.10.22

% Preamble
\documentclass[11pt]{article}

% Packages
\usepackage{amsmath}
\usepackage[english]{babel}
\usepackage{url}
\usepackage{libertine}
\usepackage{libertinust1math}
\usepackage[T1]{fontenc}
\usepackage{hyperref}
\usepackage{glossaries}
\usepackage[a4paper, total={6in, 9in}]{geometry}

%Glossary
\makeglossaries
\newglossaryentry{EO}{
    name=EO,
    description={An OOP paradigm}
}

\newglossaryentry{OOP}{
    name=OOP,
    description={A programming paradigm that is centered around objects}
}

\newglossaryentry{API}{
    name=API,
    description={An application programming interface
    allows two applications or a programmer and a library to talk to each other},
    plural={APIs}
}

\title{Elegant Objects: A OOP Paradigm}
\author{Florian Weingartshofer}
\date{November 23, 2022}

% Document
\begin{document}
    \selectlanguage{english}
    \maketitle


    \section{Introduction}\label{sec:motivation}
    Elegant objects (\gls{EO}) is an object-orientated programming (\gls{OOP}) paradigm with some extreme viewpoints on traditional methods.
    For example, null references or static methods are taboo.
    Renouncing some traditional software development methods should achieve a cleaner, more maintainable, and more readable codebase.
    \Gls{EO} also forces the programmer to think purely in objects and \gls{OOP} and not other programming paradigms, like procedural.


    \section{Content of the Paper}\label{sec:content-of-the-paper}
    The paper will first describe the twelve principles of \gls{EO} and show how they work by using some examples of the project that is developed while writing the paper.

    Furthermore, the paper will conclude how and if the codebase is more maintainable.

    Lastly, the paper will describe how the developer experience changes by using \gls{EO}.
    The most important aspects are:
    \begin{itemize}
        \item Developer productivity
        \item Code readability
        \item Ease of understanding \gls{EO} concepts
    \end{itemize}


    \section{Project}\label{sec:project}
    The project is an implementation of google guava\footnote{\url{https://github.com/google/guava}}.
    Not all guava features will be implemented, but some selected examples, that will show how \gls{EO} influences the code structure and patterns that are used.
    Some aspects of guava that can be implemented are:
    \begin{itemize}
        \item Eventbus
        \item Ranges
        \item String utils
    \end{itemize}


    \section{Organization}\label{sec:organization}
    This and the following documents will be written in English.

    \printglossary

\end{document}